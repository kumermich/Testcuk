%%%%%%%%%%%%%%%%%%%%%%%%
%
% $Author: Sadegh Naderi $
% $Datum: 2023-08-15  $
% $Pfad: BA23-14-Packages\report\Contents\General\en\pdf2txt.tex $
% $Version: 6.0 $
% $Reviewed by: Sadegh Naderi $
% $Review Date: 2023-09-13 $
%
%%%%%%%%%%%%%%%%%%%%%%%%


\chapter{Package \PYTHON{pdf2txt}}


\section{Introduction}

The package \PYTHON{pdf2txt} is a powerful and versatile tool designed to extract text content from PDF documents. In a world where information is often locked away in PDF files, \PYTHON{pdf2txt} comes to the rescue by providing a straightforward and efficient way to convert PDFs into easily accessible and manipulatable text data. Whether you're dealing with research papers, legal documents, reports, or any other PDF-based content, \PYTHON{pdf2txt} empowers you to unlock the textual information contained within these files, enabling further analysis, search, and processing. \cite{Fenniak:2008}

Please be aware that the utilized edition of the package \PYTHON{pdf2txt} in this context is \textbf{0.7.14}, accompanied by Python version \textbf{3.11.4}.


\section{Description}

The \texttt{pdf2txt} package serves as a bridge between the complex structure of PDF documents and the simplicity of text-based data. It offers a streamlined process to extract text from PDFs while maintaining the formatting, layout, and structure of the original document. By utilizing this package, you can effortlessly convert PDF files into plain text files or other text-based formats, making it easier to integrate the extracted information into your data pipelines or applications.

\subsection{Key features}

Key features of the package \PYTHON{pdf2txt} include:

\begin{enumerate}
	\item \textbf{Text Extraction:} \PYTHON{pdf2txt} excels at extracting text content from a variety of PDF files, ranging from simple text-based documents to more complex ones with images, tables, and intricate layouts.
	
	\item \textbf{Layout Preservation:} The package retains the original layout of the document, ensuring that the extracted text maintains its context and structure. This is particularly useful when dealing with documents that contain important formatting, such as legal contracts or academic papers.
	
	\item \textbf{Customization:} While extracting text is the primary function, \texttt{pdf2txt} offers customization options to control how text is parsed, such as specifying the page range, handling headers and footers, and managing special characters.
	
	\item \textbf{Integration:} The extracted text can be seamlessly integrated into your data analysis workflows, text mining projects, or even used for creating accessible versions of PDF content.
	
	\item \textbf{Automation:} For large-scale operations, \PYTHON{pdf2txt} can be scripted to automate the process of converting multiple PDF files into text format, saving time and effort.
	
	\item \textbf{Command-Line Interface (CLI):} The package includes a command-line interface, allowing users to convert PDFs to text without writing extensive code. This is particularly useful for quick conversions or when working with batch processing.
\end{enumerate}


\section{Installing pdf2txt}

\subsection{Python version support}

\texttt{pdf2txt} requires \texttt{Python} version 3.6 or higher.

\subsection{Required dependencies}

The package \PYTHON{pdf2txt} relies on several dependencies to function properly. These dependencies include:

\begin{itemize}
	\item \texttt{pandas}
	\item \texttt{pdf2image}
	\item \texttt{pdfminer.six}
	\item \texttt{Pillow}
\end{itemize}

Fortunately, when installing the \texttt{pdf2txt} package using either \texttt{pip} or \texttt{conda}, the installation process will automatically take care of installing these dependencies for you. This ensures that the \texttt{pdf2txt} package can work seamlessly without any manual intervention required on your part.

\subsection{Installing using pip}

\begin{enumerate}[label=Step \arabic*:, leftmargin=*]
	\item \textbf{Prerequisites:} Before installing `pdf2txt`, make sure you have Python installed on your system. You can download Python from the official website: \texttt{https://www.python.org/downloads/}
	
	\item \textbf{Setting Up a Virtual Environment (Recommended):} Using a virtual environment helps isolate project dependencies. Follow these steps:
	
	\begin{itemize}
		\item For Windows:
		\begin{verbatim}
			python -m venv mypdfenv
			mypdfenv\Scripts\activate
		\end{verbatim}
		
		\item For macOS and Linux:
		\begin{verbatim}
			python3 -m venv mypdfenv
			source mypdfenv/bin/activate
		\end{verbatim}
	\end{itemize}
	
	\item \textbf{Installing pdf2txt:} With the virtual environment activated, install `pdf2txt` using \texttt{pip}, the Python package manager:
	
	\begin{verbatim}
		pip install pdf2txt
	\end{verbatim}
	
	\item \textbf{Deactivating the Virtual Environment:} When done, deactivate the virtual environment:
	
	\begin{verbatim}
		deactivate
	\end{verbatim}
\end{enumerate}

\subsection{Installing using anaconda}

\begin{enumerate}[label=Step \arabic*:, leftmargin=*]
	\item \textbf{Install Anaconda:} If you haven't already, download and install Anaconda from the official website: \\
	\texttt{https://www.anaconda.com/products/distribution}
	
	\item \textbf{Open Anaconda Prompt:} Search for "Anaconda Prompt" in your system's applications and open it.
	
	\item \textbf{Create a Virtual Environment (Recommended):} In the Anaconda Prompt, run the following command to create a new virtual environment named \texttt{mypdfenv}:
	
	\begin{verbatim}
		conda create --name mypdfenv
	\end{verbatim}
	
	\item \textbf{Activate the Virtual Environment:} Activate the virtual environment using the following command:
	
	\begin{verbatim}
		conda activate mypdfenv
	\end{verbatim}
	
	\item \textbf{Installing pdf2txt:} With the virtual environment activated, install `pdf2txt` using the \texttt{conda} package manager:
	
	\begin{verbatim}
		conda install -c conda-forge pdf2txt
	\end{verbatim}
	
	\item \textbf{Deactivating the Virtual Environment:} When you're done working on your project, deactivate the virtual environment:
	
	\begin{verbatim}
		conda deactivate
	\end{verbatim}
\end{enumerate}


\section{Python example code}

\subsection{Input file}

The input file is a one page CV downloaded from the \URL{www.coolfreecv.com} website. It is shown in the Figure \ref{fig:CV}.

\begin{figure}[h!]
	\centering
	\includegraphics[width=.7\textwidth]{Images/pdf2txt/CVinput.png}
	\caption{The CV input file} \label{fig:CV}
\end{figure}

\subsection{Python code}

This code uses the pdf2txt package to extract text from each page of a PDF file and stores the extracted text in a dictionary, which is then displayed on the console.

Import the \PYTHON{pdf2txt} package as shown in Listing \ref{code:import}.

\begin{code}[h!]
	\lstinputlisting[language=Python, numbers=none, linerange={11-12}]{../Code/General/pdf2txt/pdf2txtIntroduction.py}    
	
	\caption{Importing the \PYTHON{pdf2txt} package}
	\label{code:import}
\end{code}

This imports the \PYTHON{pdf2txt} package, which you'll use to work with PDF documents.

Define the \PYTHON{extract\_text\_from\_pdf} function (see Listing \ref{code:function}).

\begin{code}[h!]
	\lstinputlisting[language=Python, numbers=none, linerange={13-16}]{../Code/General/pdf2txt/pdf2txtIntroduction.py}    
	
	\caption{Defining the \PYTHON{extract\_text\_from\_pdf} function}
	\label{code:function}
\end{code}

This function takes a PDF file path as an input and tries to extract text from it using the \PYTHON{pdf2txt} package. It starts by creating a \PYTHON{PdfDocument} instance using the provided PDF file path.

Initialize an empty dictionary to store the extracted text as shown in Listing \ref{code:dictionary}.

\begin{code}[h!]
	\lstinputlisting[language=Python, numbers=none, linerange={19-20}]{../Code/General/pdf2txt/pdf2txtIntroduction.py}    
	
	\caption{Initializing an empty dictionary}
	\label{code:dictionary}
\end{code}

This dictionary will store the extracted text from each page of the PDF.

Iterate through each page and extract text (Listing \ref{code:iterate}).

\begin{code}[h!]
	\lstinputlisting[language=Python, numbers=none, linerange={22-25}]{../Code/General/pdf2txt/pdf2txtIntroduction.py}    
	
	\caption{Iterating through each page and extracting text}
	\label{code:iterate}
\end{code}

This loop iterates through each page of the PDF using the \PYTHON{pdf\_document.pages} attribute. For each page, it uses the \PYTHON{extract\_text()} method to extract the text content. The extracted text is then added to the \PYTHON{extracted\_text\_dict} dictionary with the page number as the key.

Return the extracted text dictionary as shown in Listing \ref{code:returnDict}.

\begin{code}[h!]
	\lstinputlisting[language=Python, numbers=none, linerange={27-30}]{../Code/General/pdf2txt/pdf2txtIntroduction.py}    
	
	\caption{Returning the extracted text dictionary}
	\label{code:returnDict}
\end{code}

If the text extraction process encounters an error, it prints an error message and returns \texttt{None}. Otherwise, it returns the dictionary containing the extracted text.

Specify the path to the PDF file as shown in Listing \ref{code:path}.

\begin{code}[h!]
	\lstinputlisting[language=Python, numbers=none, linerange={33-34}]{../Code/General/pdf2txt/pdf2txtIntroduction.py}    
	
	\caption{Specifying the path to the PDF file}
	\label{code:path}
\end{code}

Call the function to extract text (see Listing \ref{code:call}):

\begin{code}[h!]
	\lstinputlisting[language=Python, numbers=none, linerange={36-37}]{../Code/General/pdf2txt/pdf2txtIntroduction.py}    
	
	\caption{Calling the function to extract text}
	\label{code:call}
\end{code}

This line calls the \PYTHON{extract\_text\_from\_pdf} function with the specified PDF file path and stores the extracted text dictionary in the \PYTHON{extracted\_text\_dict} variable.

Save the extracted text in \texttt{extractedText.txt} file as shown in Listing \ref{code:save}.

\begin{code}[h!]
	\lstinputlisting[language=Python, numbers=none, linerange={39-46}]{../Code/General/pdf2txt/pdf2txtIntroduction.py}    
	
	\caption{Saing the extracted text in \texttt{extractedText.txt} file}
	\label{code:save}
\end{code}

This code block checks if the text extraction was successful. If so, it iterates through the dictionary and exports the extracted text for each page in the \texttt{extractedText.txt} file. If extraction failed, it prints a message indicating that text extraction failed.

The complete code is shown in Listing \ref{code:exampleCodepdf2txt}.

\begin{code}[h!]
	\lstinputlisting[language=Python, numbers=none, linerange={11-}]{../Code/General/pdf2txt/pdf2txtIntroduction.py}    
	
	\caption{The Python example code using the pdf2txt package}
	\label{code:exampleCodepdf2txt}
\end{code}

\subsection{Output and the results}

The output of the Python example code in Listing \ref{code:exampleCodepdf2txt} which is saved in the \texttt{extractedText.txt} file is shown here:

\begin{verbatim}[h!]
	Page Page 1:
	Contact  
	+1 (970) 343  888 999 
	george.evans@gmail.com  
	https://www.coolfreecv.com  
	32 ELM STREET MADISON, SD 
	57042  
	George  Evans  
	PHP / OOP   
	Zend Framework  Summary  
	Senior Web Developer specializing in front end development . 
	•
	•
	•   
	Bachelor of Science: Computer Information Systems  - 2018  
	Columbia University, NY  
	
	Certifications  
	PHP Framework (certificate): Zend, Codeigniter, Symfony. 
	Programming Languages: JavaScript, HTML5, PHP OOP, CSS, SQL, 
	MySQL.  
	Reference  
	Adam Smith - Luna Web Design  
	adam.smith@luna.com  +1(970 )555 555  Skills   
	JavaScript   Symfony Framework
\end{verbatim}

The displayed text represents the output generated by the provided code. Please be aware that only a segment of the text is presented here due to spatial limitations. However, there is still room for improvement. By employing regular expressions, the text can be parsed more effectively, yielding a dictionary with organized keys and their corresponding values. Note that only a part of the text is shown here.


\section{Further Reading}

"Text Mining and Visualization: Case Studies Using Open-Source Tools" \cite{Hofmann:2016} presents a practical and illuminating exploration of text mining through the lens of open-source tools. This book takes readers on a journey into the world of text data analysis, demonstrating the power and versatility of open-source software like R and Python. Through a series of insightful case studies, the authors showcase how these tools can be harnessed to extract valuable insights from unstructured text data. Readers are offered a hands-on experience, as the book walks them through real-world examples, providing practical guidance on preprocessing, analysis, and visualization of textual information.

"A Brief Survey of Text Mining" \cite{Hotho:2005} presents a concise yet comprehensive overview of the field of text mining. The paper delves into key text mining techniques, including text preprocessing, document representation, and various mining tasks such as clustering, classification, and information extraction. Furthermore, it offers insights into the challenges and future directions in text mining, providing a solid foundation for those seeking to understand the evolving landscape of text analysis and its significance in modern information retrieval and knowledge discovery.

"Text Processing in Python" \cite{Mertz:2003} is a comprehensive guide that delves into the realm of text manipulation using the Python programming language. This book equips readers with practical insights and techniques to efficiently handle text data, from string manipulation and regular expressions to working with text files and more complex text-related tasks. With clear explanations and illustrative examples, the book empowers readers to harness Python's capabilities for text processing, making it an essential resource for programmers, data analysts, and anyone seeking to effectively manage and manipulate textual information using Python."

The article "pdf2table: Extracting Table Info from PDFs"  \cite{Yildiz:2005} presents a method to efficiently extract table information from PDFs. The authors use heuristics to recognize and break down tables in PDFs, capturing structure and content. Extracted data is stored in XML format for easy reuse. They offer a prototype for adjusting extracted data and highlight the effectiveness of heuristic-based approaches, particularly for well-structured tables. The work addresses challenges in table formats and underscores the importance of efficient data extraction in the age of abundant accessible information.

The article "A Benchmark and Evaluation for Text Extraction from PDF" \cite{Bast:2017} discusses the difficulty of extracting meaningful text from PDFs due to their layout-based format. The authors create a benchmark of 12,098 scientific articles from arXiv.org using TeX and PDF data, and evaluate 14 text extraction tools, including their own Icecite method. Icecite outperforms other tools but isn't perfect. The article aims to solve the challenges of PDF text extraction and is published in the Proceedings of Joint Conference On Digital Libraries in June 2017.

