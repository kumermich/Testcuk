\chapter{Package \PYTHON{PyPDFium2}} %https://pypdfium2.readthedocs.io/en/stable/readme.html
\section{Description}

\begin{itemize}
    \item PyPDFium2 is a Python library designed for extracting text, images, and metadata from PDF files. It provides developers with a versatile tool for comprehensive PDF data extraction in Python applications \cite{pypdfium2:2024}.
\end{itemize}

\section{Key Features}

The key features include \cite{pypdfium2:2024}: 
\begin{itemize}
    \item Text Extraction: PyPDFium2 excels in extracting textual content from PDF files, making it a valuable tool for various text-based applications.
    \item Image Extraction: Capable of extracting images embedded in PDF documents, expanding the range of data that can be retrieved.
    \item Metadata Extraction: Extracts metadata information from PDF files, providing additional insights into the document's properties.
    \item Status: Active and well-maintained, with regular updates and contributions from the community.
    \item Cross-Platform Compatibility: The Python library "PyPDFium2" is not a widely known or commonly used library, and there is limited information available about its compatibility.
    \begin{itemize}
    	\item Python Compatibility: 
    	\item PDF Versions: 
    	\item PDF Types:  Works only with machine-generated PDF, not on Scanned PDF and has limited functionalities on OCR-processed PDF. 
    \end{itemize}
    \item Output format: text string for text exraction and for image extraction, images can be extracted in many image formats.
    \end{itemize}

\section{Installation}

The installation can be done as follows \cite{pypdfium2:2024}: 
\begin{itemize}
    \item pip: 
    \item[] \SHELL{pip install PyPDFium2}
    \item conda: 
    \item[] \SHELL{conda config --add channels bblanchon}
    \item[] \SHELL{conda config --add channels pypdfium2-team}
    \item[] \SHELL{conda config --set channel\_priority strict}
    \item[] \SHELL{conda install pypdfium2-team::pypdfium2\_helpers}
\end{itemize}

\section{Getting Started}

After installation, you can integrate PyPDFium2 into your Python scripts for extracting text, images, and metadata from PDFs. The text can be extracted as follows \cite{pypdfium2:2024}: 
\begin{itemize}
    \item[] \PYTHON{import pypdfium2 as pdfium}
    \item[] \PYTHON{pdf = pdfium.PdfDocument("./path/to/document.pdf")}
    \item[] \PYTHON{version = pdf.get\_version()  \# get the PDF standard version}
    \item[] \PYTHON{n\_pages = len(pdf)  \# get the number of pages in the document}
    \item[] \PYTHON{page = pdf[0]  \# load a page}
    \item[] \PYTHON{textpage = page.get\_textpage() \# Load a text page helper}
    \item[] \PYTHON{text\_all = textpage.get\_text\_range() \# Extract text from the whole page}
\end{itemize}

\section{Further Resources}

\begin{itemize}
    \item PyPDFium2 GitHub Repository: \href{https://github.com/pypdfium2-team/pypdfium2}{pypdfium2}
    \item For more details and advanced usage of PyPDFium2, refer to the PyPDFium2 PyPI page and the official GitHub repository. \href{https://pypi.org/project/pypdfium2/}{pypdfium2py}
\end{itemize}
