\chapter{Package \PYTHON{tabula-py}}

\section{Description}

tabula-py is a versatile Python package specifically designed to extract tables from PDF documents and convert them into pandas DataFrames. This functionality facilitates data analysis by enabling users to efficiently manipulate and analyze tabular data from PDFs using the robust pandas library. \cite{Ariga:2024}. Acting as a wrapper for tabula-java, tabula-py requires Java to be installed on the user's machine. \cite{Ariga:2024}. This versatile package offers compatibility with various output formats, including CSV, TSV, and JSON, making it a valuable tool for data extraction and analysis.\cite{Ariga:2024}.


\section{Key Features}

Following are the key features \cite{Ariga:2024}.

\begin{itemize}
    \item PDF extraction accuracy comparable to tabula-java and the tabula app's GUI tool.
    \item Automation capabilities with \PYTHON{Python} scripts.
    \item Advanced analytics facilitated through the conversion of pandas DataFrames.
    \item Ideal for casual analytics using Jupyter notebooks or Google Colab.
    \item Supports multiple output formats, including CSV, TSV, JSON, and more.
    \item Status: Active and well-maintained, with regular updates and contributions from the community.
    \item Cross-Platform Compatibility: 
    \begin{itemize}
    \item Python Compatibility: It is designed to work with Python 3.x versions.
    \item Java Compatibility: Requires a Java environment on the user's machine.
    \item PDF Versions: PDFs that contain tables, regardless of the specific PDF version.
    \item PDF Types: Works with both machine generated and scanned PDF but the effectiveness might be limited in case of scanned PDFs. 
    \end{itemize}
    \item Output format: Output is pandas DataFrame which can be further converted into CSV, Excel, JSON, TSV etc., based on your needs.
\end{itemize}

\section{Installation}

To use tabula-py, follow these steps \cite{Ariga:2024}:

Check the Java environment on your machine:
\begin{itemize}
\item[] \SHELL{!java -version}
\end{itemize}
Install tabula-py using pip:
\begin{itemize}
\item[] \SHELL{!pip install -q tabula-py}
\end{itemize}

\subsection{Getting Started}
Before utilizing tabula-py, confirm your environment details using the function \PYTHON{environment\_info()} \cite{Ariga:2024}:
\begin{itemize}
\item[] \PYTHON{import tabula}
\item[] \PYTHON{tabula.environment\_info()}
\end{itemize}

\begin{itemize}
\item Code to Extract with DataFrame as Output:
\begin{itemize}
\item[] \PYTHON{Extract all pages}
\item[] \PYTHON{tabula.read\_pdf(pdf\_path, pages="all", stream=True)}
\end{itemize}

\item Extract to JSON:
\begin{itemize}
\item[] \PYTHON{Read PDF as JSON}
\item[] \PYTHON{tabula.read\_pdf(pdf\_path, output\_format="json")}
\end{itemize}

\item Convert PDF Tables into CSV:
\begin{itemize}
\item[] \PYTHON{Convert from PDF into CSV}
\item[] \PYTHON{tabula.convert\_into(pdf\_path, "test.csv", output\_format="csv", stream=True)}
\item[] \PYTHON{!cat test.csv}
\end{itemize}

\item Use Lattice Mode for More Accurate Extraction for Spreadsheet-Style Tables:
\begin{itemize}
\item[] \PYTHON{Use lattice mode for tables with lines separating cells}
\item[] \PYTHON{tabula.read\_pdf(pdf\_path, lattice=True, pages="all", stream=True)}
\end{itemize}
\end{itemize}

\section{Further Resources}

More information can be found on the following \cite{Ariga:2024}: 
\begin{itemize}
    \item GitHub Repository: \href{https://github.com/chezou/tabula-py}{tabula-py}.
    \item Example Notebook: \href{https://nbviewer.org/github/chezou/tabula-py/blob/master/examples/tabula\_example.ipynb}{tabula\_example.ipynb}.
\end{itemize}