%%%
%
% $Autor: Wings $
% $Datum: 2021-05-14 $
% $Dateiname: 
% $Version: 4620 $
%
% !TeX spellcheck = GB
% !TeX program = pdflatex
% !TeX encoding = utf8
%
%%%


\chapter{Package \PYTHON{PyTest}}
\index{PyTest}

\section{Introduction}
\index{PyTest!Introduction}
PyTest is a popular testing framework for Python, known for its simplicity and powerful features. It allows writing simple unit tests as well as complex functional tests.

\section{Description}
\index{PyTest!Description}
PyTest provides an easy way to write, organize, and run tests using Python. It supports fixtures, parameterized testing, and powerful assertions, making testing efficient and scalable.

\section{Installation}
\index{PyTest!Installation}
To install PyTest, use the following command:

\begin{verbatim}
pip install pytest
\end{verbatim}

You can verify the installation with:

\begin{verbatim}
pytest --version
\end{verbatim}

\section{Example - Writing a Simple Test}

Create a Python file named \texttt{test\_example.py} and add the following code:

\begin{lstlisting}[language=Python, caption=Basic PyTest Example]
"""
@file test_example.py
@brief This file contains a basic test case using PyTest.
@date \today
"""

def test_addition():
    """Test that addition works correctly."""
    assert 1 + 1 == 2

# Run the test with:
# pytest test_example.py
\end{lstlisting}

To execute the test, open a terminal and run:

\begin{verbatim}
pytest test_example.py
\end{verbatim}

\section{Example - Using Fixtures}
Fixtures allow setting up test preconditions and dependencies. Here's an example:

\begin{lstlisting}[language=Python, caption=Using Fixtures in PyTest]
"""
@file test_fixtures.py
@brief Demonstrates the use of fixtures in PyTest.
@date \today
"""

import pytest

@pytest.fixture
def sample_data():
    """Provides sample data for test cases."""
    return [1, 2, 3, 4, 5]

def test_length(sample_data):
    """Test that the sample data has the correct length."""
    assert len(sample_data) == 5
\end{lstlisting}

\section{Example - Parameterized Testing}
Parameterized tests allow running a test function with multiple sets of inputs. Example:

\begin{lstlisting}[language=Python, caption=Parameterized Test in PyTest]
"""
@file test_parametrize.py
@brief Demonstrates parameterized testing in PyTest.
@date \today
"""

import pytest

@pytest.mark.parametrize("a, b, expected", [
    (1, 2, 3),
    (4, 5, 9),
    (10, 15, 25)
])
def test_addition(a, b, expected):
    """Test that addition works correctly with different inputs."""
    assert a + b == expected
\end{lstlisting}

\section{Example - Checking Test Coverage}
To check test coverage, install `pytest-cov`:

\begin{verbatim}
pip install pytest-cov
\end{verbatim}

Run tests with coverage:

\begin{verbatim}
pytest --cov=my_package
\end{verbatim}

\section{Further Reading}
\begin{itemize}
    \item Official PyTest Documentation: \href{https://docs.pytest.org/}{here}
    \item Real Python's PyTest Guide: \href{https://realpython.com/pytest-python-testing/}{here}
\end{itemize}

\nocite{Abadi:2016}

	% Überschrift ein Level unter `refsection=chapter`, also \section*:
   \printbibliography[heading=subbibliography, segment=\therefsegment]










