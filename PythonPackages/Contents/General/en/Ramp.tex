%%%
%
% $Autor: Wings $
% $Datum: 2021-05-14 $
% $Dateiname: 
% $Version: 4620 $
%
% !TeX spellcheck = GB
% !TeX program = pdflatex
% !TeX encoding = utf8
%
%%%


\chapter{Package \PYTHON{Ramp}}
\index{Ramp}

\section{Introduction}
\index{Ramp!Introduction}
Ramp is a Python package used for [provide description of Ramp's purpose and use case]. It simplifies [describe functionality briefly].

\section{Description}
\index{Ramp!Description}
Ramp provides an easy-to-use interface for [describe key features and functionalities]. It supports [list some notable capabilities].

\section{Installation}
\index{Ramp!Installation}
To install Ramp, use the following command:

\begin{verbatim}
pip install ramp
\end{verbatim}

You can verify the installation with:

\begin{verbatim}
pip show ramp
\end{verbatim}

\section{Example - Manual Usage}

Describe how to manually use Ramp with an example:

\begin{lstlisting}[language=Python, caption=Basic Ramp Example]
"""
@file example_ramp.py
@brief This script demonstrates basic usage of the Ramp package.
@date \today
@details This script imports the Ramp package and calls a sample function.
"""

import ramp

# Example usage
def main():
    """Main function to execute Ramp functionality."""
    result = ramp.function_name(parameters)
    print(result)

if __name__ == "__main__":
    main()
\end{lstlisting}

\section{Example}
Provide another example related to a specific use case of Ramp.

\section{Example - Version}
To check the installed version of Ramp, use:

\begin{verbatim}
pip show ramp | grep Version
\end{verbatim}

\section{Example - Files}
A sample Ramp Python script:

\begin{verbatim}
PackageExample.py
\end{verbatim}

\section{Further Reading}
\begin{itemize}
   \item Official Ramp Documentation: \href{https://ramp.readthedocs.io/en/latest/}{here}
   \item Example Ramp Project on GitHub: \href{https://github.com/RAMP-project/RAMP}{here}
\end{itemize}


\nocite{Abadi:2016}

	% Überschrift ein Level unter `refsection=chapter`, also \section*:
   \printbibliography[heading=subbibliography, segment=\therefsegment]










