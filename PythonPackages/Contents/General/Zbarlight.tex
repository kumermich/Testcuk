%%%%%%%%%%%%%%%
%
% $Autor: Wings $
% $Datum: 2020-02-24 14:30:26Z $
% $Pfad: PythonPackages/Contents/General/zbarlight.tex $
% $Version: 1792 $
%
% !TeX encoding = utf8
% !TeX root = PythonPackages
% !TeX TXS-program:bibliography = txs:///bibtex
%
%
%%%%%%%%%%%%%%%



\chapter{Package \PYTHON{zbarlight}}

\section{Introduction}

zbarlight is a simple wrapper for the zbar package. For now, it can read all zbar supported codes. Contributions, suggestions and pull requests are welcome.\cite{zbarlightpypi:2024}

\section{Description}

\PYTHON{zbarlight} is a lightweight Python package that acts as a wrapper for the ZBar barcode and QR code scanner. ZBar is a versatile package capable of decoding various barcode types from images or video streams. \textbf{zbarlight} simplifies the process of integrating ZBar’s powerful barcode decoding functionalities into Python applications, focusing on ease of use and efficient image decoding. \cite{zbarlightpypi:2024}

\section {Key Features of zbarlight}

\begin{itemize}
	\item \textbf{Barcode and QR Code Decoding:} Supports multiple barcode formats, including QR codes, EAN, UPC, and more.
	\item \textbf{Lightweight and Fast:} Designed to be minimalistic with quick processing for individual or batch image decoding.
	\item \textbf{Seamless Integration:} Works well with Python’s popular \textit{Pillow} package (formerly \textit{PIL}) for image processing.
	\item \textbf{Cross-Platform:} While dependent on the ZBar package, it can be set up on Linux, macOS, and Windows with appropriate installation steps.
\end{itemize}


\section{Installation}

You can install zbarlight as follows \cite{zbarlightpypi:2024}: 

\subsection{Prerequisites}
Ensure the following dependencies are met before proceeding with the installation:
\begin{itemize}
	\item \textbf{ZBar package}: \texttt{zbarlight} depends on the ZBar package.
	\item \textbf{Pillow package}: Required for handling images in Python.
\end{itemize}

\subsection{Installing the ZBar package}

\subsection{For Debian-based Linux}
Run the following commands to install the ZBar package:
\begin{lstlisting}[language=bash]
	sudo apt-get update
	sudo apt-get install libzbar0 libzbar-dev
\end{lstlisting}

\subsection{For macOS}
Install ZBar using \texttt{brew}:
\begin{lstlisting}[language=bash]
	brew install zbar
\end{lstlisting}

Set the environment variables required for compiling \texttt{zbarlight}:
\begin{lstlisting}[language=bash]
	export LDFLAGS="-L$(brew --prefix zbar)/lib"
	export CFLAGS="-I$(brew --prefix zbar)/include"
\end{lstlisting}

\subsection {For Windows}
1. Download the ZBar binaries for Windows from \href{https://github.com/mchehab/zbar}{ZBar GitHub}.
2. Add the ZBar binary directory to your system's PATH.
3. Ensure your Python version matches the ZBar binary compatibility.

\subsection{Installing the zbarlight Package}
Once the ZBar package is installed, install \texttt{zbarlight} using \texttt{pip}:
\begin{lstlisting}[language=bash]
	pip install zbarlight
\end{lstlisting}

\subsection{Verifying the Installation}
To verify the installation, run the following Python script:
\begin{lstlisting}[language=Python]
	import zbarlight
	print("zbarlight installed successfully!")
\end{lstlisting}
If no errors occur, \texttt{zbarlight} is installed correctly.

\subsection{Installing Pillow}
If the Pillow package is not already installed, you can install it using:
\begin{lstlisting}[language=bash]
	pip install pillow
\end{lstlisting}

\subsection{Troubleshooting}
\begin{itemize}
	\item \textbf{ZBar package Not Found}: Ensure ZBar is installed and added to the system PATH.
	\item \textbf{Environment Variables on macOS}: Use \texttt{brew} to locate ZBar package paths and export them correctly.
	\item \textbf{Pillow Compatibility}: Upgrade Pillow if necessary using:
	\begin{lstlisting}[language=bash]
		pip install pillow --upgrade
	\end{lstlisting}
	\item \textbf{Windows-specific Issues}: Verify that ZBar binaries are properly installed and accessible via PATH.
\end{itemize}


\section{Getting Started}

A simple example of how to use zbarlight to decode QR codes from an image is as follows \cite{zbarlightpypi:2024}: 

\begin{lstlisting}[language=Python]
from PIL import Image
import zbarlight

file_path = './tests/fixtures/two_qr_codes.png'
with open(file_path, 'rb') as image_file:
image = Image.open(image_file)
image.load()

codes = zbarlight.scan_codes(['qrcode'], image)
print('QR codes: %s' % codes)
\end{lstlisting}

\section{Example - Basic Usage of Zbarlight}
The core functionality of the 
\PYTHON{zbarlight} package revolves around the \PYTHON{scan\_codes()} function. This function is responsible for decoding barcodes or QR codes from an image. Here's how it works:\\

\begin{lstlisting}[language=Python]
	zbarlight.scan_codes(symbols, image)
\end{lstlisting}

\subsection{Supported Barcode Formats}
\texttt{zbarlight} supports a wide range of barcode and QR code formats. Commonly used formats include:

\begin{itemize}
	\item \textbf{EAN-13}: Common in retail products.
	\item \textbf{EAN-8}: A shortened version of EAN-13 for small products.
	\item \textbf{UPC-A}: Used in North American retail systems.
	\item \textbf{Code-128}: High-density barcode often used in logistics.
	\item \textbf{Interleaved 2 of 5}: Used for shipping and warehouse management.
\end{itemize}

\subsection{Reading Images Using \texttt{Pillow}}
The \texttt{zbarlight} package requires image input in the form of \texttt{Pillow} (Python Imaging package) objects. \texttt{Pillow} simplifies reading and preprocessing images for barcode or QR code decoding.

Here’s how to load an image using \texttt{Pillow}:
\begin{lstlisting}[language=Python]
	from PIL import Image
	
	# Load the image file
	with open('example.png', 'rb') as image_file:
	image = Image.open(image_file)
	image.load()
\end{lstlisting}

The \PYTHON{Image.open()} function opens the image, and \PYTHON{image.load()} ensures the image is ready for further processing.

\begin{lstlisting}[language=Python]
	from PIL import Image
	import zbarlight
	
	# Open the image containing the QR code
	with open('qrcode.png', 'rb') as image_file:
	image = Image.open(image_file)
	image.load()
	
	# Decode the QR code
	codes = zbarlight.scan_codes(['qrcode'], image)
	print('Decoded QR Code:', codes)
\end{lstlisting}


\section{Example - Advanced Usage of Zbarlight}

\subsection{Working with Multi-Barcode/QR Code Images}
In scenarios where an image contains multiple barcodes or QR codes, you can use libraries like \texttt{zbarlight} to detect all codes in one scan. The \PYTHON{scan\_codes()} function can return a list of all detected codes.\cite{zbarlightgithub:2024}

\textbf{Example:}
\begin{lstlisting}[language=Python]
	from PIL import Image
	import zbarlight
	
	# Load image with multiple QR codes
	with open('multi_qrcodes.png', 'rb') as image_file:
	image = Image.open(image_file)
	image.load()
	
	codes = zbarlight.scan_codes(['qrcode'], image)
	print('Detected QR Codes:', codes)
\end{lstlisting}

\subsection{Batch Processing Multiple Images}
For applications like inventory management or document scanning, we might need to decode barcodes/QR codes from multiple images at once. Batch processing involves looping through a directory of images and applying the decoding logic.\cite{zbarlightgithub:2024}

\begin{lstlisting}[language=Python]
	import os
	from PIL import Image
	import zbarlight
	
	# Folder containing images
	image_folder = 'images/'
	for image_file in os.listdir(image_folder):
	with open(os.path.join(image_folder, image_file), 'rb') as file:
	image = Image.open(file)
	image.load()
	codes = zbarlight.scan_codes(['qrcode'], image)
	print(f'{image_file}: {codes}')
\end{lstlisting}

\subsection{Handling Rotated or Skewed Barcodes}
Sometimes barcodes or QR codes may not be perfectly aligned. Libraries like \texttt{zbarlight} automatically handle minor rotations, but significant skew may require preprocessing. Rotating the image to different angles before scanning can help.\cite{zbarlightgithub:2024}

\begin{lstlisting}[language=Python]
	from PIL import Image
	import zbarlight
	
	# Rotate the image before decoding
	with open('skewed_qrcode.png', 'rb') as image_file:
	image = Image.open(image_file)
	rotated_image = image.rotate(45)  # Rotate by 45 degrees
	codes = zbarlight.scan_codes(['qrcode'], rotated_image)
	print('Decoded Code:', codes)
\end{lstlisting}

\section{Integration with Other Libraries}
Zbarlight is a lightweight and efficient package designed for decoding barcodes and QR codes in Python. One of its standout features is its flexibility in integrating with other libraries to enhance functionality. For instance, Zbarlight can be combined with popular image processing libraries like Pillow (PIL) to preprocess images before barcode decoding. By leveraging Pillow’s capabilities, developers can easily manipulate image properties such as resizing, cropping, or adjusting brightness and contrast, ensuring better decoding accuracy.\cite{zbarlightpypi:2024} \\

In addition, Zbarlight can be paired with OpenCV for advanced use cases like real-time barcode detection in video streams. OpenCV can handle video capture and image frame extraction, while Zbarlight decodes the barcodes within these frames. This integration proves especially useful in applications like inventory management systems, automated checkout systems, and logistics tracking.\cite{opencv_documentation:2024} \\

Moreover, the simplicity of Zbarlight allows it to seamlessly work with web frameworks such as Flask or Django. Developers can build web-based barcode scanning tools by integrating Zbarlight to process user-uploaded images and return decoded information. \\

\section{Error Handling and Limitations}
Zbarlight primarily relies on the Zbar package for decoding, and its error-handling mechanisms are somewhat basic. If a barcode or QR code cannot be detected, the package simply returns None without providing detailed error information. This can make debugging challenging in scenarios where the image quality is poor or the barcode is not properly aligned. Additionally, Zbarlight does not natively support logging or descriptive error messages, leaving developers to implement their own mechanisms to handle such situations effectively. \cite{zbarlightgithub:2024}\\

Another drawback is its limited support for modern barcode formats. While ZBar primarily focuses on common 1D and 2D barcodes, it lacks support for emerging formats, which could be a hindrance in advanced applications. Furthermore, Zbarlight doesn’t provide detailed error messages, which can make debugging difficult for developers.\\

\section{Further Resources}

\textit{zbarlight on PyPI} \cite{zbarlightpypi:2024}: This page provides an overview of the \texttt{zbarlight} package, including its features, installation instructions, and compatibility details. It serves as the primary distribution source, allowing developers to install the package via \texttt{pip}. The page also highlights the package's dependencies and Python version requirements, making it a useful reference for setting up \texttt{zbarlight} in Python projects.\\

\textit{zbarlight GitHub Repository} \cite{zbarlightgithub:2024}: The official GitHub repository for \texttt{zbarlight} hosted by Polyconseil contains the source code, documentation, and examples for using the package. It is a valuable resource for developers who want to explore the codebase, report issues, or contribute enhancements. The repository also includes details on setting up the required environment, usage examples, and troubleshooting tips for common issues.\\

\textit{Getting zbarlight to Work on Windows} \cite{Dhambarage:2020}: This Medium article provides a comprehensive guide to configuring \texttt{zbarlight} on Windows, which can be challenging due to its dependencies on the ZBar package. The tutorial outlines the steps to install necessary libraries, troubleshoot installation errors, and ensure compatibility with the Python environment.\\
