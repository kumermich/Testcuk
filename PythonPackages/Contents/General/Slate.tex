%%%%%%%%%%%%%%%
%
% $Autor: Abishek $
% $Datum: 2025-01-29 14:30:26Z $
% $Pfad: PythonPackages/Contents/General/slate.tex $
% $Version: 1792 $
%
% !TeX encoding = utf8
% !TeX root = PythonPackages
% !TeX TXS-program:bibliography = txs:///bibtex
%
%
%%%%%%%%%%%%%%%

\chapter{Package \PYTHON{Slate}} %https://pypi.org/project/slate/

\section{Introduction}

Slate is a Python library that primarily focuses on text extraction from PDFs. It provides developers with a specialized tool for efficiently extracting textual content from PDF documents \cite{slate:2015}.

\section{Description}
Slate is a Python package that makes it easy to extract text from PDF files. It relies on another package called PDFMiner to do this. Slate provides a simple class called \texttt{PDF} that you can use to read text from each page of a PDF document.

\begin{lstlisting}[language=Python, caption=Basic Text Extraction Using Slate]
	with open('example.pdf') as f:
	doc = slate.PDF(f)
	# Access the extracted text
	doc
	['Text from page 1...', 'Text from page 2...', ...]
	# Example: Access text from page 2
	doc[1]
	'Text from page 2...'
\end{lstlisting}

If your PDF file has a password, you can include the password when creating the \texttt{PDF} object:
\begin{lstlisting}[language=Python, caption=Text Extraction from Password-Protected PDFs]
	with open('secrets.pdf') as f:
	doc = slate.PDF(f, 'password')
	# Access the extracted text
	doc[0]
	"My mother doesn't know this, but..."
\end{lstlisting}
If you need to access more than just the text—like images or font information—you will need to learn how to use the PDFMiner API directly. Slate is designed for simplicity and focuses only on extracting text.
\begin{figure}[h!]
	\centering
	\begin{tikzpicture}[
		node distance=1cm and 2cm,
		every node/.style={draw, rounded corners, align=center},
		process/.style={rectangle, fill=red!20, text width=3.5cm},
		intermediate/.style={rectangle, fill=red!20, text width=3.5cm},
		result/.style={rectangle, fill=blue!20, text width=3.5cm}
		]
		
		% Nodes
		\node[process] (pdf) {PDF File};
		\node[process, right=of pdf] (slate) {Slate Package};
		\node[intermediate, below=of slate] (pdfminer) {PDFMiner};
		\node[result, below=of pdfminer] (text) {Extracted Text};
		
		% Arrows
		\draw[->, thick] (pdf) -- (slate) ;
		\draw[->, thick] (slate) -- (pdfminer) ;
		\draw[->, thick] (pdfminer) -- (text) ;
		
	\end{tikzpicture}
	\caption{Slate Utilizing PDFMiner.}
	\label{fig:pdf_slate_pdfminer}
\end{figure}

\section{Key Features}

\begin{itemize}
    \item Text Extraction: Slate excels in extracting textual content from PDF files, making it a valuable tool for applications where text data is the primary focus.
    \item PDF Processing: The library is tailored for processing PDF files and aims to streamline the extraction of text, recognizing the importance of text-based information in PDF documents.
    \item It depends on the PDFMiner package. If we would like access to the images, font files and other information, we would have to depend on PDFMiner API.  
    \item Status: Last Commit: About 7 years ago (as of my last knowledge update in January 2022). Status: Potentially inactive or incomplete. A lack of recent updates could indicate the project is not actively maintained.
    \item Cross-Platform Compatibility: 
    \begin{itemize}
    \item Python Compatibility: Python 2.6
    \item PDF Versions: 
    \item PDF Types: 
    \end{itemize}
    \item Output format: Output is a plain text string. 
\end{itemize}

\section{Installation}

To install Slate, you can use the pip package manager by running the following command in your terminal:

\begin{itemize}
    \item[] \SHELL{pip install slate}
\end{itemize}

\section{Getting Started}

After installation, you can integrate Slate into your Python scripts for extracting text from PDFs.

\begin{itemize}
    \item[] \PYTHON{import slate}
    \item[] \PYTHON{with open('example.pdf', 'rb') as file:
				doc = slate.PDF(file)}
    \item[] \PYTHON{print(doc[1])}
\end{itemize}

\section{Example - Description}
This example demonstrates the functionality of the Slate package by testing its ability to process PDF files. The focus is on verifying features such as text extraction, metadata retrieval, and handling password-protected PDF files. It utilizes the \texttt{pytest} framework to structure and automate the testing process.

\section{Example - Manual}
The testing includes the following processes:
\subsection{Test Workflow Overview}
The test script utilizes the Slate package to process two types of PDFs:
\begin{itemize}
	\item \textbf{Non-password-protected PDF:} Ensures basic functionality of text extraction and metadata handling.
	\item \textbf{Password-protected PDF:} Verifies that the package can decrypt and extract data from encrypted files.
\end{itemize}

\subsubsection{Testing Workflow}
The testing workflow involves:
\begin{enumerate}
	\item \textbf{Loading PDF files:} Files are loaded using the \texttt{PDF} class.
	\item \textbf{Extracting text and metadata:} Text content and associated metadata are retrieved for validation.
	\item \textbf{Testing text outputs:} Both cleaned and uncleaned text outputs are tested to ensure versatility.
	\item \textbf{Validating password-protected files:} Ensures proper decryption and data extraction from encrypted PDFs.
\end{enumerate}

\noindent All tests are designed to be executed using the \texttt{pytest} framework for structured and efficient testing.

\section{Example - Code}
\subsection{Test Script for Slate}
\begin{lstlisting}[language=Python, caption=Test Script for Slate]
	"""
	Tests for slate
	http://pypi.python.org/slate
	
	Expected to be used with py.test:
	http://codespeak.net/py/dist/test/index.html
	"""
	
	from classes import PDF
	
	# Function to return a non-password-protected PDF object
	def pytest_funcarg__doc(request):
	with open('example.pdf', 'rb') as f:
	return PDF(f)
	
	# Function to return a password-protected PDF object
	def pytest_funcarg__passwd(request):
	with open('protected.pdf') as f:
	return PDF(f, 'a')
	
	# Test case to verify basic text extraction
	def test_basic(doc):
	assert doc[0] == 'This is a test.\n\n\x0c'
	
	# Test case to verify metadata extraction
	def test_metadata_extraction(doc):
	assert doc.metadata
	
	# Test case to verify text extraction via the `text()` method
	def test_text_method(doc):
	assert "This is a test" in doc.text()
	
	# Test case to verify uncleaned text extraction
	def test_text_method_unclean(doc):
	assert '\x0c' in doc.text(clean=0)
	
	# Test case to verify text extraction from a password-protected PDF
	def test_password(passwd):
	assert passwd[0] == "Chamber of secrets.\n\n\x0c"
\end{lstlisting}

\section{Key Sections of the Code}

\subsection{1. Importing Dependencies}
\begin{lstlisting}[language=Python]
	from classes import PDF
\end{lstlisting}
\textbf{Purpose:} Imports the \texttt{PDF} class for parsing and extracting text from PDF files.

\subsection{2. Handling Non-Password-Protected PDFs}
\begin{lstlisting}[language=Python]
	def pytest_funcarg__doc(request):
	with open('example.pdf', 'rb') as f:
	return PDF(f)
\end{lstlisting}
\textbf{Purpose:} Loads a standard PDF file and returns it as a \texttt{PDF} object.

\subsection{3. Handling Password-Protected PDFs}
\begin{lstlisting}[language=Python]
	def pytest_funcarg__passwd(request):
	with open('protected.pdf') as f:
	return PDF(f, 'a')
\end{lstlisting}
\textbf{Purpose:} Loads a password-protected PDF file with decryption support.

\subsection{4. Basic Text Extraction}
\begin{lstlisting}[language=Python]
	def test_basic(doc):
	assert doc[0] == 'This is a test.\n\n\x0c'
\end{lstlisting}
\textbf{Purpose:} Ensures accurate text extraction from the first page.

\subsection{5. Metadata Extraction}
\begin{lstlisting}[language=Python]
	def test_metadata_extraction(doc):
	assert doc.metadata
\end{lstlisting}
\textbf{Purpose:} Verifies that metadata (e.g., title, author) can be retrieved from the PDF.

\subsection{6. Cleaned and Uncleaned Text Extraction}
\begin{lstlisting}[language=Python]
	def test_text_method(doc):
	assert "This is a test" in doc.text()
	
	def test_text_method_unclean(doc):
	assert '\x0c' in doc.text(clean=0)
\end{lstlisting}
\textbf{Purpose:} Validates the ability to extract text in both cleaned and raw formats.

\subsection{7. Password-Protected PDF Test}
\begin{lstlisting}[language=Python]
	def test_password(passwd):
	assert passwd[0] == "Chamber of secrets.\n\n\x0c"
\end{lstlisting}
\textbf{Purpose:} Confirms successful decryption and text extraction from encrypted PDFs.

\section{Example - Files}
\subsection{Relevant Files in the Directory}
\begin{itemize}
	\item \textbf{example.pdf:} 
	\begin{itemize}
		\item A non-password-protected test PDF file.
		\item Contains the text: \texttt{"This is a test."}.
	\end{itemize}
	\item \textbf{protected.pdf:} 
	\begin{itemize}
		\item A password-protected test PDF file.
		\item Content: \texttt{"Chamber of secrets."} with the password: \texttt{a}.
	\end{itemize}
	\item \textbf{test\_slate.py:} 
	\begin{itemize}
		\item The test script file located in the directory.
		\item Contains the Python code used to validate the Slate package.
	\end{itemize}
\end{itemize}

\subsection{Directory Structure}
\begin{verbatim}
	.
	+-- src/slate/
	|   +-- __init__.py        # Initializes the Slate package
	|   +-- classes.py         # Contains the PDF class for processing PDF files
	|   +-- conf_test.py       # Configuration file for pytest framework
	|   +-- example.pdf        # Non-password-protected test PDF file
	|   +-- protected.pdf      # Password-protected test PDF file
	|   +-- test_slate.py      # Test script to validate the Slate package functionality
	|   +-- unittests/         # Directory containing additional unit tests
	|   -- utils.py            # Utility functions for the Slate package
	README.txt                 # Documentation for the Slate package
\end{verbatim}


\section{Further Reading}
For more information, refer to the following resources:
\begin{itemize}
	\item \textbf{Slate PyPI Page:} \url{http://pypi.python.org/slate}
	\item \textbf{Pytest Documentation:} \url{http://codespeak.net/py/dist/test/index.html}
	\item \textbf{Slate GitHub Repository:} \url{https://github.com/timClicks/slate}
\end{itemize}

\section{Further Resources}

\begin{itemize}
    \item Slate GitHub Repository: \href{https://github.com/timClicks/slate}{slate}
    \item For more details and advanced usage of Slate, refer to the Slate PyPI page and the official GitHub repository.
\end{itemize}
